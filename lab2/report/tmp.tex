\documentclass[UTF8]{ctexart}
\usepackage{ctex}
\usepackage{amsmath}
\usepackage{graphicx}
\usepackage{listings}
\usepackage{xcolor} 
\usepackage[colorlinks,linkcolor=red]{hyperref}
\usepackage{colortbl,booktabs}
\bibliographystyle{acm}
\title{预备工作2:定义你的编译器\&汇编编程}
\author{冯杰康,计算机科学专业,1813402}
\date{2020年10月}
\lstset{
    basicstyle=\ttfamily,
    language=C++,
    numbers=left, 
    numberstyle= \tiny, 
    keywordstyle= \color{ blue!60!green!45!},
    commentstyle= \color{blue!90!},
    tabsize=4,   
    frame=shadowbox, % 阴影效果
    numbersep=3pt,
    rulesepcolor= \color{ red!20!green!20!blue!20} ,
    escapeinside=``, % 英文分号中可写入中文
    xleftmargin=2em,xrightmargin=2em, aboveskip=1em,
    backgroundcolor=\color[RGB]{245,245,244},
    framexleftmargin=2em
}
\begin{document}
\maketitle
\begin{abstract}
    在本次实验的工作中,我们将要大体上设计我们所需要的编译器的语法结构,并且我们会设计
    一部分CFG来描述我们的编译器,并且使其靠近C语言或者他的子集。这些语言特征我们也将在
    之后的编译器实现上进行部分实现。
    为了更加了解高级语言编译到汇编程序的过程,我们也会在本次共实验中将一个尽可能完整的
    程序手工编译成汇编过程,之后详细解释了其中的过程,并且提升其普适性,将在其中使用
    CFG和语法制导翻译等技术,并且进行汇编验证。
\end{abstract}
{\bf{关键词:编译器;汇编;CFG;}}
\section{引言}
    我们的编译器设计将在C语言的基础上进行设计,主要语法支持将为C语言的子集,会在部分语法文法上参考SysY系统进行
    设计。SysY中进行\cite{SysY}了部分语言的概要设计,并且其在比赛中也取得了不错的成绩,我们的编译
    系统也将要主要参考其进行设计。

    在一个高级语言的编译成可执行程序的时候,会经过预处理器、编译器、汇编器、链接器加载器等过程。在最重要的编译器过程中,
    将会经过词法分析、语法分析、语义分析、中间代码生成、代码优化、代码生成的过程,最终生成汇编代码。我们会进行一个将
    C语言手工翻译成汇编代码的实验,并尝试进行跑通测试,总结近似的编译规律。
\section{我的工作}
实验主要分为两部分,第一部分为编译器的定义和设计,第二部分为汇编程序编程。
\subsection{编译器定义}
我们所了解的C语言支持很多语言特征,并且很多也有多种方式的支持,语言灵活度较大。
\begin{itemize}
    \item 多种数据类型的支持,如int,double等;
    \item 循环、分支语句的支持,并且对于也有多种写法的支持,while和for,if和switch等,在新的c++11标准\cite{c11}中也支持了range—for等用法;
    \item 变量的声明支持,其中也有常量,静态变量等多种声明方式的支持;
    \item 多种运算的支持,位运算和加减乘除取模,并且在进行除法是也有不同的整除和浮点除法的不同支持,在其中也有类型转换等多种判断,但是并没有支持乘方(幂次)操作;
    \item 逻辑运算的支持,有与或非异或,其他的逻辑运算可以用这些方式进行组合,并且区分位运算和逻辑运算的处理方式;关系运算的支持,对于大于小于等关系运算的支持,虽然详尽但是并不支持连续的关系运算,只能通过
关系运算和逻辑运算两个组合进行处理,如"2<=a<=5"需要转化成"2<=a \&\& a<=5"进行运算;
    \item 函数的调用机制,支持变量参数的传递,支持递归调用等方式,但是返回值有所限制,只能返回基本类型;
    \item 数组和指针的语法支持,这也是在编码过程中比较复杂的一部分,由于C语言并不具备引用这种功能,因此指针传递
是在函数传参和返回值上面重要应用位置,并且支持连续内存的定长数组,指针也可以指向函数结构体等。
\end{itemize}
    这些语法功能综合完成了最原始的C语言结构,C语言的标准在也不断扩充不断完善,有新的语法特征和新的语法功能加入其中,使得其更加方便安全,我们的工作
    编译器也将完成其中的一个子集,之后进行不断扩充完善功能。

\subsubsection{我的编译器}
    在我们所要支持的编译器中,我们需要支持C语言的一个子集,完成最小的一个高级语言的编译工作。

    我们计划支持一下语法结构,并且大部分将参考已经实现过的SysY\cite{SysY}的语法体系:
    \begin{itemize}
        \item 数据类型支持int,以及其支持的数组,并且在其中我们也将对多维数组采取行优先连续的方式存储;
        \item 变量声明我们支持普通的声明以及常量声明;
        \item 函数的支持,可以返回int值或者空置,也可以传入参数,支持传入数组参数和int参数;
        \item 循环、分支语句的支持可以完成if语句和while语句的识别,并且在其中实现break和continue等特殊功能语句的实现,并且只其语句块内支持变量定义和语句嵌套等功能;
        \item 支持算数运算,逻辑运算,位运算等运算方式,可以使用1代替true,使用0代替false;
        \item 赋值语句和一般的声明赋初值等的支持;
    \end{itemize}
    以上的语法特点已经可以完成一个比较小的编译器了,并且在以上的支持之中还可以进行扩充。不断地扩充过程也是不断地
    完善编译器的过程,从C语言的子集不断发展,以上便是我们设计的编译器所需要完成的基本任务。
\subsubsection{CFG设计}
    对于以上的自定义编译器,为了更快速的完成以后的工作,我们在此设计了其CFG,大部分参开于SysY\cite{SysY},并且部分文法可以进行后续扩充,完善
    编译器工作。

    我们参考SysY并为之设计的CFG如下:
    终结符为Ident,InstConst等符号,或者直接由引号引起
    \begin{itemize}
        \item 编译单元 CompUnit$\rightarrow$(CompUnit$\mid \epsilon$)(Decl $\mid$ FuncDef)
        \item 基本类型: Btype $\rightarrow$ 'int'
        \item 声明:Decl $\rightarrow$ ConstDecl $\mid$ VarDecl
        \item 常量声明:ConstDecl  $\rightarrow$ 'const' Btype ConstDef (',' ConstDef)$^*$';'
        \item 常量初值:ConstDef $\rightarrow$ ConstExp $\mid$ '\{'(ConstInitVal(','ConstInitVal)$^*$)$\mid \epsilon$'\}'
        \item 常数定义:ConstDef $\rightarrow$ {\bf{Ident}} ('['ConstExp']')$^*$'='ConstOnitVal
        \item 变量声明:ValDecl $\rightarrow$ BType VarDef (','Vardef)$^*$';'
        \item 变量初值:InitVal$\rightarrow$Exp$\mid$'\{'(InitVal(','InitVal)$^*$)$\mid \epsilon$'\}'
        \item 变量定义:FuncDef $\rightarrow$ {\bf{Ident}}('['ConstExp']')$^*$ $\mid$ {\bf{Ident}}('['ConstExp']')$^*$'='InitVal
        \item 函数定义:FuncDef  $\rightarrow$ FuncType {\bf{Ident}} '(' (FuncParams$\mid \epsilon$) ')'Block
        \item 语句块:Block $\rightarrow$ '\{'  $BlockItem^{*}$ '\}'
        \item 函数类型: FuncType$\rightarrow$ 'void'$\mid$'int'
        \item 函数形参列表:FuncParams$\rightarrow$ FuncParam(',' FuncParam)$^*$
        \item 函数形参:FuncParam $\rightarrow$ BType {\bf{Ident}} '(' ('['']'('['Exp']')$^*$ $\mid \epsilon$)
        \item 语句块项:BlockItem$\rightarrow$Decl$\mid$Stmt
        \item 语句
         Stmt $\rightarrow$ LVal'='Exp';'$\mid$(Exp$\mid \epsilon$';') $\mid$ Block $\mid$ 
        'if' '('Cond')' Stmt (('else'Stmt) $\mid \epsilon$) $\mid$ 'while''('Cond')'Stmt $\mid$ 'break'';'$\mid$'continue'';'
        $\mid$'return'(Exp$\mid \epsilon$)';'
        \item 表达式:Exp$\rightarrow$AddExp
        \item 条件表达式:Cond $\rightarrow$ LOrExp
        \item 左值表达式:LVal $\rightarrow${\bf{Ident}}(('['Exp']')$\mid \epsilon$)
        \item 基本表达式:PrimaryExp$\rightarrow$'('Exp')' $\mid$ LVal $\mid$ Number
        \item 数值:Number$\rightarrow${\bf{IntConst}}
        \item 一元表达式:UnaryExp$\rightarrow$Primary$\mid${\bf{Ident}}'('(FuncRParams$\mid \epsilon$)')'$\mid$UnaryOp UnaryExp
        \item 单目运算符:UnaryOp$\rightarrow$'+'$\mid$'-'$\mid$'!'
        \item 函数实参表:FuncRParams$\rightarrow$Exp(','Exp)$^*$
        \item 乘除模表达式:MulExp$\rightarrow$UnaryExp$\mid$MulExp('*'$\mid$'/'$\mid$'\%')UnaryExp
        \item 加减表达式:AddExp$\rightarrow$MulExp$\mid$AddExp('+'$\mid$'-')MulExp
        \item 关系表达式:RelExp$\rightarrow$AddExp$\mid$RelExp('<'$\mid$'>'$\mid$'<='$\mid$'>=')AddExp
        \item 相等性表达式:EqExp$\rightarrow$RelExp$\mid$EqExp('=='$\mid$'!=')RelExp
        \item 逻辑与表达式:LAndExp$\rightarrow$EqExp$\mid$LAndExp'\&\&'EqExp
        \item 逻辑或表达式:LOrExp$\rightarrow$LAndExp$\mid$LOrExp'||'EqExp
        \item 常量表达式:ConstExp$\rightarrow$AddExp
    \end{itemize}
    以上参考SysY的语法定义系统,我们也计划在其中增加部分其他语法。
    如do-while语法:Stmt$\rightarrow$'do' '\{'Stmt'\}' 'while' '(' Cond ')',还有如三元运算符:
    CalTree$\rightarrow$ Cond '?' Stmt ':' Stmt等等,都可以为后续的工作进行拓展实现。
\subsection{汇编编程}
为了提现尽可能多的语法特征,我们采用了计算斐波那契数列的代码进行汇编翻译工作,这样可以体现出C语言编译为汇编的大部分过程。
C语言源代码如下:
\begin{lstlisting}
    #include<stdio.h>
    int input;
    int a=0,b=1;
    int i=1;
    int result;
    int Fibonacci(int n)
    {
        while(i<n)
        {
            result=a+b;
            a=b;
            b=result;
            i++;
        }
        return result;
    }
    
    int main()
    {
        scanf("%d",&input);
        printf("%d",Fibonacci(input));
        return 0;
    }
\end{lstlisting}
这是一个简易的计算斐波那契数列的小程序,我们在之后将其手工编译成汇编代码。

在对比指导手册代码和部分gcc编译的汇编代码之后,我们将其手工编程成汇编程序如下:
\begin{lstlisting}
    .text
    .globl Fibonacci
    .type Fibonacci,@function
#全局变量存储
	.comm	input, 4, 2
	.globl	a
	.bss
	.align 4
a:
	.space 4
	.globl	b
	.data
	.align 4
b:
	.long	1
	.globl	i
	.align 4
i:
	.long	1
	.comm	result, 4, 2
    .section .rodata


STR0:
    .string "%d"
STR1:
    .string "%d"
#函数

L3:
    movl a, %ecx
    movl b, %edx
    addl %ecx, %edx
    movl %edx, result
    movl b, %eax
    movl %eax, a
    movl result, %eax
    movl %eax, b
    movl i, %eax
    addl $1, %eax
    movl %eax, i

Fibonacci:
    movl 4(%esp), %eax
    movl i, %ebx
    cmpl  %ebx, %eax
    jg L3
# return result
    movl result, %eax
    ret






     .text
     .globl main
     .type main, @function

main:
#输入数字
    pushl $input
    pushl $STR0
    call scanf
    addl $8, %esp
#输出函数,并且调用的函数参数压栈
    movl input, %edx
    pushl %edx
    call Fibonacci
    addl $4, %esp
# %eax保存返回值  
    pushl %eax 
    pushl $STR1
    call printf
    addl $8, %esp
# return 0
    movl $0, %eax
    ret
    .section .note.GNU-stack,"",@progbits
\end{lstlisting}
在使用qemu硬件模拟之后我们也得到了正确的结果。
\section{结论及后续工作}
本次实验中我们完成了我们的编译器的语法定义,并且进行的CFG的设计,在之后的工作中我们会进一步完善整个编译器,并且尽量实现编译器的扩充,
如在其中加入double类型,加入switch和for支持等等,甚至对于指针等动作的支持。

在汇编工作中,我们完成了32位汇编的工作,更好的了解的汇编和C语言源代码的对应过程。
\bibliography{ref}
\end{document}